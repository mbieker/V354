\documentclass[11pt]{article}
%Gummi|061|=)
\usepackage{amsmath}
\usepackage{amsthm}
\usepackage{amsbsy}
\usepackage{amssymb}
\usepackage{inputenc}
\usepackage{graphicx}
\usepackage{selinput}
\SelectInputMappings{%
adieresis={ä},
germandbls={ß},
}
\title{\textbf{Versuch V354: Gedämpfte und erzwungene Schwingungen}}
\author{Martin Bieker\\
		Julian Surmann\\
		\\
		Durchgef\"{u}hrt am 19.12.2013\\
		Tu Dortmund}
\date{}
\usepackage{graphicx}
\begin{document}
\renewcommand\tablename{Tabelle}
\renewcommand\figurename{Abbildung}
\maketitle
\thispagestyle{empty}
\newpage
\clearpage
\setcounter{page}{1}


\section{Einleitung}
In diesem Versuch sollen gedämpfte und erzwungene Schwingungen am Beispiel des elektrischen Schwingkreises untersucht werden. Dieser sogenannte RCL-Schwingkreis besteht aus zwei Energiespeichern, einem Kondensator und einer Spule.
\section{Theorie}
Hier Theorie einsetzen, Formeln mit:
\begin{equation}
F O R M E L  
\end{equation}
\section{Aufbau und Durchf\"{u}hrung}
Hier folgt der Aufbau und die Durchführung.
\section{Auswertung}
Hier folgt die Auswertung.
Das ist eine geeignete Tabelle mit 5 Spalten:
\begin{table}[h]
\centering
\begin{tabular}{|c|c|c|c|c|}
\hline
$\varphi[^\circ]$ & $\varphi [rad]$ & $F [N]$ & $M [N]$ & $ D [\frac{Nm}{rad}]$ \\
\hline
20.0 & 0.349 & 0.1 & 0.01 & 0.0286\\
40.0 & 0.698 & 0.2 & 0.02 & 0.0286\\
\hline
\end{tabular}
\caption{Benennung der Tabelle}
\end{table}
\section{Diskussion}
Hier kommt die Diskussion hin.
\section{Literatur- und Abbildungsverzeichnis}
Hier befindet sich das Literatur- und Abbildungsverzeichnis.
\section{Anhang}
Hier stehen die im Anhang angefügten Dokumente.
\end{document}
