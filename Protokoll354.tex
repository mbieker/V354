\documentclass[11pt,ngerman,a4paper]{article}
%Gummi|061|=)
\usepackage{amsmath}
\usepackage{a4wide}
\usepackage{amsthm}
\usepackage{amsbsy}
\usepackage{amssymb}
\usepackage{inputenc}
\usepackage{rotating} 
\usepackage{graphicx}
\usepackage{paralist}
\usepackage{selinput}
\SelectInputMappings{%
adieresis={ä},
germandbls={ß},
}
\title{\textbf{Versuch V354: Gedämpfte und erzwungene Schwingungen}}
\author{Martin Bieker\\
		Julian Surmann\\
		\\
		Durchgef\"{u}hrt am 19.12.2013\\
		Tu Dortmund}
\date{}
\usepackage{graphicx}
\begin{document}
\renewcommand\tablename{Tabelle}
\renewcommand\figurename{Abbildung}
\maketitle
\thispagestyle{empty}
\newpage
\clearpage
\setcounter{page}{1}


\section{Einleitung}
In diesem Versuch sollen gedämpfte und erzwungene Schwingungen am Beispiel des elektrischen Schwingkreises untersucht werden. Dieses System besteht aus zwei Energiespeichern, einem Kondensator und einer Spule. Zwischen diesen Energiespeichern pendelt die Energie hin und her. Unter (theoretischen) idealen Bedingungen ist diese Schwingung ungedämpft. In dem zu untersuchenden Versuchsaufbau ist jedoch ein Widerstand eingebaut, so wird eine gedämpfte Schwingung beobachtet.
\section{Theorie}
Bei dem Versuchsaufbau handelt es sich um einen gedämpften harmonischen Oszillator. Mit Hilfe der Kirchhoffschen Regeln und dem Induktionsgesetz kann die Differentialgleichung des hier verwendeten Oszillators hergeleitet werden. Sie lautet
\begin{equation}
\label{DGL}
\frac{d^2 I}{d t^2} + \frac{R}{L} \frac{d I}{d t} + \frac{I}{LC} = 0.
\end{equation}
Um diese Differentialgleichung zu lösen, ist eine Fallunterscheidung von Nöten.\newline
Für den Schwingfall lautet die Lösung
\begin{equation}
\label{LSG1}
I(t) = A_0 e^{-2 \pi \mu t} cos (2 \pi \nu t + \eta).
\end{equation}
Dabei ist die Schwingung nach der Zeit
\begin{equation}
\label{Tex}
T_{ex} = \frac{2L}{R}
\end{equation}
auf den $e$-ten der Amplitude abgeklungen.
Für den aperiodischen Grenzfall lautet die Lösung
\begin{equation}
\label{LSG2}
I(t) = A e^{- \frac{t}{\sqrt{LC}}}.
\end{equation}
Um einen theoretischen Dämpfungswiderstand am aperiodischen Grenzfall zu berechnen, wird
\begin{equation}
\label{aper}
R_{ApThe} = \sqrt{\frac{4L}{C}}
\end{equation} verwendet.
Der theoretische Wert einer Resonanzfrequenz ist durch
\begin{equation}
f_res = \frac1{2\pi} \sqrt{\frac{1}{LC}-\frac{R^2}{2L^2}}
\label{frequenz}
\end{equation}
Die Güte eines Schwingkreises kann mit
\begin{equation}
q = \frac{1}{R} \sqrt{\frac{L}{C}}
\label{gute}
\end{equation}
errechnet werden.
Die Breite einer Resonanzkurve ist mit
\begin{equation}
f_+ - f_- \approx \frac{R}{L}
\label{breite}
\end{equation}
gegeben.
\section{Aufbau und Durchf\"{u}hrung}
Die Schaltung des elektrischen Schwingkreises ist fest aufgebaut. Es stehen mehrere Widerstände zur Verfügung, davon ein Potentiometer. Ein Signalgenerator wird benutzt, um das System anzuregen. Für die Messungen wird ein digitales Speicheroszilloskop verwendet.
\subsection{Untersuchung der Zeitabhängigkeit der Amplitude}
Um die Zeitabhängigkeit der Amplitude zu untersuchen, wird der elektrische Schwingkreis mit einzelnen Nadelimpulsen angeregt. Das System beginnt zu Schwingen. Mit dem hochohmigen Tastkopf wird die Spannung des Kondensators auf die Y-Ablenkung des Speicheroszilloskops gegeben. Der entstehende Graph sowie die Messwerte sollen für die Auswertung gespeichert werden. Bei diesem Versuchsteil ist der kleinere Widerstand zu verwenden. Der Versuchsaufbau dieses Versuchsteiles ist in Abbildung \ref{S1} zu sehen.
\begin{figure}[h]
\centering
\includegraphics[scale=0.7]{Aufbau1.png}
\caption{Schaltung zur Messung der Zeitabhängigkeit der Amplitude $[1]$}
\label{S1}
\end{figure}
\subsection{Bestimmung des Dämpfungswiderstandes bei aperiodischem Grenzfall}
In diesem Versuchsteil soll der Dämpfungswiderstand $R_{ap}$, bei dem die Schwingung den aperiodische Grenzfall annimmt, ermittelt werden. Dazu wird in der Schaltung aus Abbildung \ref{S1} lediglich der konstante Widerstand gegen das Potentiometer ausgetauscht. Das Potentiometer wird nun in maximale Stellung gebracht, der Kriechfall sollte zu erkennen sein. Nun wird der Widerstand solange verändert, bis man zwischen dem Kriech- und Schwingfall den aperiodischen Grenzfall möglichst gut angenähert hat.
\subsection{Messung der Frequenzabhängigkeit der Kondensatorspannung}
In dieser Messung wird die Frequenz des Erregers verändert und dabei die Auswirkung auf die maximale Kondensatorspannung untersucht. Da der verwendete Tastkopf auf einen spezifischen Frequenzgang besitzt, muss auch die Erregerspannung gemessen werden. In der Auswertung wird dann der Quotient $U_C / U_{Er}$ ermittelt.
Es sollen mindestens zehn Messwerte genommen werden. Für eine bessere Auswertung werden in der Nähe des Maximums der Kondensatorspannung Messwerte mit geringem Abstand aufgezeichnet, während die Messwerte außerhalb dieses Bereiches weiter von einander entfernt sind.
Dieser Versuchsteil wird mit Hilfe der Schaltung in Abbildung \ref{S2} durchgeführt. Dabei soll der größere konstante Widerstand verwendet werden.
\begin{figure}[h]
\centering
\includegraphics[scale=0.7]{Aufbau2.png}
\caption{Schaltung zur Messung der Frequenzabhängigkeit von $U_C$ $[1]$}
\label{S2}
\end{figure}
\subsection{Bestimmung der Frequenzabhängigkeit der Phase}
Zur Bestimmung der Phasendifferenz zwischen der Generator- und der Kondensatorspannung werden beide Spannungen auf die Y-Achse des Speicheroszilloskops gelegt. Bei einer Phasenverschiebung mit $\varphi \neq 0$ sind die beiden Kurven so gut zu erkennen. Die Phasenverschiebung wird dann mit Hilfe der Nulldurchgänge der Schwingungen ermittelt. Dafür müssen die beiden Sinuskurven allerdings symmetrisch zur x-Achse liegen. Für eine genauere Auswertung werden die gesamten Daten gespeichert und mit Algorithmen ausgewertet werden, anstatt die Werte einfach abzulesen. Die verwendete Schaltung ist der Abbildung \ref{S3} zu entnehmen.
\begin{figure}[h]
\centering
\includegraphics[scale=0.7]{Aufbau3.png}
\caption{Schaltung zur Bestimmung der Frequenzabhängigkeit der Phase $[1]$}
\label{S3}
\end{figure}



\section{Auswertung}
Alle Fehler wurden, sofern nicht anders angegeben, mit python uncertainties ermittelt.
\subsection{Bestimmung der Zeitkonstante}
Abbildung \ref{abb1} zeigt den Verlauf der Spannung $U_C$ zwischen den Platten des Kondensators unmittelbar nach dem Abschalten der Aufladespannung.
\begin{figure}[h!]
\centering
\includegraphics[scale=0.7]{Abb/abb1.png}
\caption{Spannung des Kondensators nach Abschalten der Aufladespannung}
\label{abb1}
\end{figure}
\begin{figure}[h!]
\centering
\includegraphics[scale=0.7]{Abb/abb2.png}
\caption{Halblogarithmische Darstellung von Abb \ref{abb1}}
\label{abb2}
\end{figure}
In der Tabelle \ref{mtab1} befinden sich die lokalen Extrema des Spannungsverlaufs. Abbildung \ref{abb2} ist eine halb-logarithmische Darstellung dieser Werte. In diesem Diagramm liegen die Messpunkte entlang der Geraden
\begin{equation}
ln(\frac{U_C}{V}) = -\frac{1}{RC}\cdot t + ln(U_0).
\end{equation}
Durch eine lineare Regression ergeben sich folgende Werte:
\begin{itemize}
\item $\frac{1}{RC} = (-5820\pm60)\,\frac{1}{s}$
\item $T_{ex} = RC = (0.0001719\pm0.0000019)\, s$
\item $R = \frac{2L}{T_{ex}} = (117.6\pm1.3)\, \Omega$.
\end{itemize}
Der Fehler von $\frac{1}{RC}$ folgt aus der linearen Regression.
\subsection{Bestimmung des Grenzwiderstands}
Zur Bestimmung des Dämpfungswiderstands, der zum aperiodischen Grenzfall führt, wurde das Potentiometer zunächst auf sein Maximum eingestellt. Zu sehen war ein Kriechfall. Dann wurde der Widerstand vorsichtig verkleinert. Nach Erscheinen eines Überschwingens war der Widerstand bereits zu klein. So genau wie möglich wurde dann der Grenzwiderstand angenähert. Er liegt bei $(3220 \pm 20) \, \Omega$. Der Zustand der Schwingung (aperiodischer Grenzfall) ist gut in Abbildung \ref{apgrenzfall} zu erkennen. Aus den Größen L und C lässt sich mit der Formel \ref{aper} der theoretischen Wert berechnen: $(4396\pm7)\,\Omega$. Die Abweichung des experimentell bestimmten Wertes zum Theoriewert lässt sich mit der Ungenauigkeit der Messung erklären. Es ist nur grob abzuschätzen, wann der aperiodische Grenzfall eintritt. Individuelle Einstellungen am Oszilloskop erschweren dieses Problem.
\begin{figure}[h!]
\centering
\includegraphics[scale=0.5]{apgrenzfall.png}
\caption{Manuelle Einstellung des aperiodischen Grenzfalls}
\label{apgrenzfall}
\end{figure}
\subsection{Bestimmung der Resonanzfrequenz}
In Abbildung \ref{resonanz1} ist der Quotient aus der Kondensatorspannung $U_C$ und der Erregerspannung $U_0$ gegen die Erregerfrequenz $f$ aufgetragen. Das Maximum im Resonanzfall ist deutlich zu erkennen. F\"ur die Resonanzzfrequenz ergibt sich somit:
\[
	f_{res} = 33.7\,kHz.
\]
Mit Formel \ref{guete} ist der G\"uetefaktor eines Schwingkreises durch den Wert des Quotienten $\frac{U_C}{U_0}$ im Resonanzfall gegeben. Dieser betr\"agt
\[
q =  3.9
\]
Zur Bestimmung der Sch\"arfe der Resonanz wurde in dem Diagramm bei 
\[
\frac{U_C}{U_0} = \frac{q}{\sqrt{2}}
\]
eine Gerade einf\"uegt. Die Frequenzen $f_+$ und $f_-$ bei denen $\frac{U_C}{U_0}$ auf diesen Wert abgefallen ist, k\"onnen somit wie folgt bestimmt werden:
\begin{itemize}
\item $f_- = kHz$
\item $f_+ = kHz$
\end{itemize}

\noindent
Der theoretische Wert der Resonanzfrequenz ist durch Formel \ref{frequenz} durch
\[
f_res = \frac1{2\pi} \sqrt{\frac{1}{LC}-\frac{R^2}{2L^2}}
\]
gegeben. Mit den gegebenen Werten f\"ur $L$ und $C$ und dem im ersten Versuchsteil bestimmten D\"ampfungswiderstand
$R_{eff}$ ergibt sich 
\[
f_{res} = ...\,Hz
\]
Die G\"ute $q$ eines Schwingkreises kann mit Formel \ref{gute} zu
\[
q = \frac{1}{R} \sqrt{\frac{L}{C}} = ... 
\]
berechnet werden. Die Breite der Resonanzkurve sollte nach Formel \ref{breite}
\[
f_+ - f_- \approx \frac{R}{L}
\]
betragen.
\subsection{Bestimmung der Phasenverschiebung}
Im letzten Versuchsteil soll die Phasenbeziehung von Kondensatorspannung und der Erregerspannung bei verschiedenen Anregungsfrequenzen untersucht werden. Mit den Daten aus Tabelle \ref{} wird dieser Zusammenhang in Abbildung \ref{} graphisch dargestellt. Der Resonanzfall tritt bei $\Delta \varphi = 90^\circ$ auf. Die Frequenzen bei denen $\Delta \varphi$ genau $45^circ$ und $135^\circ$ ist, werden als $f_1$ und $f_2$ bezeichnet. Aus dem Diagramm wurden diese Frequenzen wie folgt bestimmt:
\begin{itemize}
\item $f_{res} = ...\,kHz$
\item $f_{1} = ...\,kHz$
\item $f_{2} = ...\,kHz$
\end{itemize}
Mit Formel \ref{} k\"onnen $f_1$ und $f_2$ auch aus den Gr\"o\ss en $R$,$L$ und $C$ berechnet werden.
\[
f_{1,2} = \frac{1}{2 \pi} (\pm \frac{R}{2L} \sqrt{R^2/4L^2 + 1/LC}
  )
\]
Es ergeben sich folgende Werte:
\begin{itemize}
\item $f_1 = ...\,kHz$
\item $f_2 = ...\,kHz$
\end{itemize}
\section{Diskussion}
In diesem Versuch ist bei den meisten Versuchsteilen eine sehr hohe Genauigkeit möglich. So sind sowohl der Signalgenerator als auch das Speicheroszilloskop digital. Da die Werte der Amplituden nicht aus einem Plot des Oszilloskops abgelesen, sondern mit einem Algorithmus aus den Rohdaten ermittelt wurden, gibt es hier keinen Ablesefehler. Die Messfühler waren hochohmig, der Fehler durch sie ist daher vernachlässigbar gering. Lediglich beim Ablesen des Widerstandes, der zum aperiodischen Grenzfalls führt, beträgt die Messungenauigkeit $\pm$ 20 Ohm.
\section{Literatur- und Abbildungsverzeichnis}
\begin{itemize}
\item $[1]$: Der Praktikumsanleitung zu V354 der TU Dortmund entnommen. Download am 5.1.14 unter \newline http://129.217.224.2/HOMEPAGE/PHYSIKER/BACHELOR/AP/SKRIPT/V354.pdf
\end{itemize}
\section{Anhang}
\begin{itemize}
\item Tabellen und Abbildungen
\item Auszug aus dem Messheft.


\end{itemize}

\newpage
\begin{table}[H]
\centering
\begin{tabular}{|c|c|c|}
\hline
$\frac{t}{ms}$ & $\frac{U_C}{V}$ & $ln(\frac{U_C}{V})$ \\
\hline
0.0046 & 18.6 & 2.923\\
0.0354 & 15.6 & 2.747\\
0.0642 & 13.2 & 2.58\\
0.0938 & 11.2 & 2.416\\
0.1234 & 9.4 & 2.241\\
0.1516 & 7.8 & 2.054\\
0.1812 & 6.6 & 1.887\\
0.211 & 5.6 & 1.723\\
0.2404 & 4.6 & 1.526\\
0.2704 & 4.0 & 1.386\\
0.3004 & 3.4 & 1.224\\
0.329 & 2.8 & 1.03\\
0.3578 & 2.2 & 0.788\\
0.3874 & 2.0 & 0.693\\
0.4184 & 1.8 & 0.588\\
0.4478 & 1.4 & 0.336\\
\hline
\end{tabular}
\label{mtab1}
\caption{Lokale Maxima des Spannungsverlaufs}
\end{table}

\end{document}
