\documentclass[11pt,ngerman,a4paper]{article}
%Gummi|061|=)
\usepackage{amsmath}
\usepackage{a4wide}
\usepackage{amsthm}
\usepackage{amsbsy}
\usepackage{amssymb}
\usepackage{inputenc}
\usepackage{rotating} 
\usepackage{graphicx}
\usepackage{paralist}
\usepackage{selinput}
\SelectInputMappings{%
adieresis={ä},
germandbls={ß},
}
\title{\textbf{Versuch V354: Gedämpfte und erzwungene Schwingungen}}
\author{Martin Bieker\\
		Julian Surmann\\
		\\
		Durchgef\"{u}hrt am 19.12.2013\\
		Tu Dortmund}
\date{}
\usepackage{graphicx}
\begin{document}
\renewcommand\tablename{Tabelle}
\renewcommand\figurename{Abbildung}
\maketitle
\thispagestyle{empty}
\newpage
\clearpage
\setcounter{page}{1}


\section{Einleitung}
In diesem Versuch sollen gedämpfte und erzwungene Schwingungen am Beispiel des elektrischen Schwingkreises untersucht werden. Dieser sogenannte RCL-Schwingkreis besteht aus zwei Energiespeichern, einem Kondensator und einer Spule.
\section{Theorie}
Hier Theorie einsetzen, Formeln mit:
\begin{equation}
F O R M E L  
\end{equation}
\section{Aufbau und Durchf\"{u}hrung}
Hier folgt der Aufbau und die Durchführung.
\section{Auswertung}
\subsection{Bestimmung der Zeitkonstante}
Abbildung \ref{abb1} zeigt den Verlauf der Spannung $U_C$ zwischen den Platten des Kondensators unmittelbar nach dem Abschalten der Aufladespannung. 

==========

In der Tabelle \ref{mtab1} finden sich die lokalen Extrema des Spannungsverlaufs. Abbildung \ref{mabb2} ist eine halb-logarithmische Darstellung dieser Werte. In diesem Diagramm liegen die Messpunkte entlang der Geraden
\begin{equation}
ln(\frac{U_C}{V}) = -\frac{1}{RC}\cdot t + ln(U_0).
\end{equation}
Durch eine lineare Regression ergeben sich folgende Werte:
\begin{itemize}
\item $\frac{1}{RC} = ....\,\frac{1}{s}$
\end{itemize}
\subsection{Bestimmung des Grenzwiderstands}
\subsection{Bestimmung der Resonanzfrequenz}
\subsection{Bestimmung der Phasenverschiebung}
\section{Diskussion}
Hier kommt die Diskussion hin.
\section{Literatur- und Abbildungsverzeichnis}
Hier befindet sich das Literatur- und Abbildungsverzeichnis.
\section{Anhang}
\begin{itemize}
\item Tabellen und Abbildungen
\end{itemize}

\newpage
\begin{table}[H]
\centering
\begin{tabular}{|c|c|c|}
\hline
$\frac{t}{ms}$ & $\frac{U_C}{V}$ & $ln(\frac{U_C}{V})$ \\
\hline
0.0046 & 18.6 & 2.923\\
0.0354 & 15.6 & 2.747\\
0.0642 & 13.2 & 2.58\\
0.0938 & 11.2 & 2.416\\
0.1234 & 9.4 & 2.241\\
0.1516 & 7.8 & 2.054\\
0.1812 & 6.6 & 1.887\\
0.211 & 5.6 & 1.723\\
0.2404 & 4.6 & 1.526\\
0.2704 & 4.0 & 1.386\\
0.3004 & 3.4 & 1.224\\
0.329 & 2.8 & 1.03\\
0.3578 & 2.2 & 0.788\\
0.3874 & 2.0 & 0.693\\
0.4184 & 1.8 & 0.588\\
0.4478 & 1.4 & 0.336\\
\hline
\end{tabular}
\label{mtab1}
\caption{Lokale Maxima des Spannungsverlaufs}
\end{table}
\end{document}
