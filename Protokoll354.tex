\documentclass[11pt,ngerman,a4paper]{article}
%Gummi|061|=)
\usepackage{amsmath}
\usepackage{a4wide}
\usepackage{amsthm}
\usepackage{amsbsy}
\usepackage{amssymb}
\usepackage{inputenc}
\usepackage{rotating} 
\usepackage{graphicx}
\usepackage{paralist}
\usepackage{selinput}
\SelectInputMappings{%
adieresis={ä},
germandbls={ß},
}
\title{\textbf{Versuch V354: Gedämpfte und erzwungene Schwingungen}}
\author{Martin Bieker\\
		Julian Surmann\\
		\\
		Durchgef\"{u}hrt am 19.12.2013\\
		Tu Dortmund}
\date{}
\usepackage{graphicx}
\begin{document}
\renewcommand\tablename{Tabelle}
\renewcommand\figurename{Abbildung}
\maketitle
\thispagestyle{empty}
\newpage
\clearpage
\setcounter{page}{1}


\section{Einleitung}
In diesem Versuch sollen gedämpfte und erzwungene Schwingungen am Bei-spiel des elektrischen Schwingkreises untersucht werden. Dieses System besteht aus zwei Energiespeichern, einem Kondensator und einer Spule. Zwischen diesen Energiespeichern pendelt die Energie hin und her. Unter (theoretischen) idealen Bedingungen ist diese Schwingung ungedämpft. In dem zu untersuchenden Versuchsaufbau ist jedoch ein Widerstand eingebaut, so kann eine gedämpfte Schwingung beobachtet werden.
\section{Theorie}
Hier Theorie einsetzen, Formeln mit:
\begin{equation}
F O R M E L  
\end{equation}
\section{Aufbau und Durchf\"{u}hrung}
\subsection{Untersuchung der Zeitabhängigkeit der Amplitude}
\subsection{Bestimmung des Dämpfungswiderstandes bei aperiodischem Grenzfall}
\subsection{Messung der Frequenzabhängigkeit der Kondensatorspannung}
\subsection{Bestimmung der Frequenzabhängigkeit der Phase}
\section{Auswertung}
\subsection{Bestimmung der Zeitkonstante}
Abbildung \ref{abb1} zeigt den Verlauf der Spannung $U_C$ zwischen den Platten des Kondensators unmittelbar nach dem Abschalten der Aufladespannung. 

==========

In der Tabelle \ref{mtab1} finden sich die lokalen Extrema des Spannungsverlaufs. Abbildung \ref{mabb2} ist eine halb-logarithmische Darstellung dieser Werte. In diesem Diagramm liegen die Messpunkte entlang der Geraden
\begin{equation}
ln(\frac{U_C}{V}) = -\frac{1}{RC}\cdot t + ln(U_0).
\end{equation}
Durch eine lineare Regression ergeben sich folgende Werte:
\begin{itemize}
\item $\frac{1}{RC} = ....\,\frac{1}{s}$
\end{itemize}
\subsection{Bestimmung des Grenzwiderstands}
\subsection{Bestimmung der Resonanzfrequenz}
\subsection{Bestimmung der Phasenverschiebung}
\section{Diskussion}
Hier kommt die Diskussion hin.
\section{Literatur- und Abbildungsverzeichnis}
Hier befindet sich das Literatur- und Abbildungsverzeichnis.
\section{Anhang}
Hier stehen die im Anhang angefügten Dokumente.
\end{document}
